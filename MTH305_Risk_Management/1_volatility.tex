\section{Volatility}
\subsection{Definition}
There are two almost same definitions of volatility vary by the expression of return.
\begin{definition}[Volatility]
    A variable's volatility, $\sigma$, is defined as the \textbf{standard deviation} of the return provided by the variable \textbf{per unit of time} when the return is expressed using \textbf{continuous compounding}.
\end{definition}
\begin{remark}[Unit of time]
    For option pricing, the unit of time is one year while for risk management, the unit of time is one day.
\end{remark}
\begin{remark}[Continuously compounded]
    Define $S_{i}$ as the value of a variable at the end of day $i$. The continuously compounded return per day for the variable on day $i$ is
    $$
        \ln \frac{S_{i}}{S_{i-1}}
    $$
\end{remark}
\noindent The term continuously compounded is almost the same as the proportional change during a day, that is
$$
\ln \frac{S_{i}}{S_{i-1}} \approx \frac{S_{i}-S_{i-1}}{S_{i-1}}
$$
\begin{definition}[Volatility]
    A variable's volatility, $\sigma$, is defined as the \textbf{standard deviation} of the return provided by the variable \textbf{per unit of time} when the return is expressed using \textbf{proportional change}.
\end{definition}
\noindent Definition 1.2 is usually used in risk management.

\subsubsection{Variance Rate and Days}
Assumption: the returns each day are independent with the same variance, time $T$, then
\begin{align*}
    \sigma_{t_0+T} &= \sqrt{T} \sigma_{t_0}\\
    \sigma_{t_0+T}^2 &= T \sigma_{t_0}^2    
\end{align*}
\begin{center}
    "uncertainty increases with the square root of time"
\end{center}
Volatility is much higher on business days than on non-business days. Hence, we using business days with 252 days per year.
$$
\sigma_{day} = \frac{\sigma_{year}}{\sqrt{252}}
$$
\subsubsection{Implied Volatilities}
The implied volatility of an option is the volatility that gives the market price of the option when it is substituted into the pricing model.

\subsection{The Power Law}
Volatility is \emph{not} constant, hence the returns on successive days are heavy tailed compared Normal distribution. The power law is more suitable in practice.
\begin{equation}
    f(x)=P(v>x)=Kx^{-\alpha} \label{eq1.1}
\end{equation}
where $K$ and $\alpha$ are constant.\myref{eq1.1} can be convert into
\begin{equation}
    \ln [\operatorname{P}(v>x)]=\ln K-\alpha \ln x
\end{equation} 
to fit real data.

\subsection{Monitoring Daily Volatility}
\paragraph{Volatility with Continuously Compounded}~{}\\
$\sigma_{n}$: the volatility per day of a market variable on day $n$, as estimated at the end of day $n-1$.\\
$S_{i}$: value of the market variable at the end of day $i$.\\
$u_{i}$: the continuously compounded return during day $i$ (between the end of day $i-1$ and the end of day $i$ ).\\
Using $m-days$' observations to monitor $\sigma_{n}$, we have
\begin{align}
    u_{i}&=\ln \frac{S_{i}}{S_{i-1}}\\
    \bar{u}&=\frac{1}{m} \sum_{i=1}^{m} u_{n-i}\\
    \sigma_{n}^{2}&=\frac{1}{m-1} \sum_{i=1}^{m}\left(u_{n-i}-\bar{u}\right)^{2}\label{eq1.2}
\end{align}
\paragraph{Volatility with Proportional Change}~{}\\
With some assumptions and tricks, we simplified \myref{eq1.2} into 
\begin{align}
    u_{i}&=\frac{S_{i}-S_{i-1}}{S_{i-1}}\\
    \bar{u}&=0\\
    \sigma_{n}^{2}&=\frac{1}{m} \sum_{i=1}^{m} u_{n-i}^{2}\label{eq1.3}
\end{align}
Different between the result of \myref{eq1.2} and \myref{eq1.3} is very small.

\subsubsection{Weighting Schemes}
\paragraph{Weighting Schemes Model}
\begin{assumption}[Weighting Schemes Model]
    More weight given to recent data.
\end{assumption}
\noindent Let $\sum_{i=1}^{m} \alpha_{i}=1$ and when $i>j$ , $\alpha_{i}>\alpha_{j}>0$, then \myref{eq1.3} converted into
\begin{align}
    \sigma_{n}^{2}=\sum_{i=1}^{m} \alpha_{i} u_{n-i}^{2}\label{eq1.4}
\end{align}
\paragraph{ARCH(m) Model}
\begin{assumption}[ARCH(m) Model]
    There is a long-run average variance rate and that this should be given some weight.
\end{assumption}
\noindent Let $V_{L}$ is the long-run variance rate and $\gamma$ is the weight assigned to $V_{L}$ and $\gamma+\sum_{i=1}^{m} \alpha_{i}=1$, then \myref{eq1.3} converted into 
\begin{align}
    \sigma_{n}^{2}&=\gamma V_{L}+\sum_{i=1}^{m} \alpha_{i} u_{n-i}^{2}\label{eq1.5}\\
    &=\omega+\sum_{i=1}^{m} \alpha_{i} u_{n-i}^{2}\notag
\end{align}

\subsubsection{EWMA Model}
