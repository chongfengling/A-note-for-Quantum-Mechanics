\documentclass[12pt, a4paper, oneside]{article}
\usepackage{amsmath, amsthm, amssymb, bm, graphicx, mathrsfs}
\usepackage{hyperref} % content with links
\usepackage{bookmark, geometry}
\geometry{a4paper,scale=0.8}

\usepackage[english]{babel} % English formatting
\usepackage{fancyhdr} % Fancy headers

\title{{\Huge{\textbf{MTH307}}}\\\Huge{\textbf{Population Dynamics}}}
\author{Chongfeng Ling}
\date{2021-2022}
\linespread{1.5}

\newtheorem{theorem}{Theorem}
\newtheorem{definition}[theorem]{定义}
\newtheorem{lemma}[theorem]{引理}
\newtheorem{corollary}[theorem]{推论}
\newtheorem{example}[theorem]{例}
\newtheorem{proposition}[theorem]{命题}
\newtheorem{assumption}{Assumption}[section]

% short commands
\newcommand{\rr}{\mathbb{R}}
\newcommand{\myref}[1]{Eq.\ref{#1}}
%~~~~~~~~~ Page setup
\pagestyle{fancy}
\fancyhf{}
\lfoot{MTH307}
\rfoot{Page \thepage}
\renewcommand{\footrulewidth}{0.2pt}
\renewcommand{\headrulewidth}{0pt}


\begin{document}

% center cover page
\null  % Empty line
\nointerlineskip  % No skip for prev line
\vfill
\let\snewpage \newpage
\let\newpage \relax
\maketitle
\setcounter{page}{0}
\thispagestyle{empty}
\let \newpage \snewpage
\vfill
\break % page break

% content page
\newpage
\pagenumbering{Roman}
\setcounter{page}{1}
\tableofcontents


\renewcommand\thepart{\Alph{part}}

\newpage
\setcounter{page}{1}
\pagenumbering{arabic}


\section{Basic Population Dynamics}

\paragraph{Difference between Population Dynamics and other models}

\paragraph{Types of models}

\paragraph{Single Species Model}~{\\
$t$ be time, changing with intervals $\tau$ (e.g. 1 year).\\
$N(t)$ be total population of the species at time $t$.\\
$\tilde{B}(t, \tau)$ be number of births in the population between $t$ and $t+\tau$, fertility/recruitment term.\\
$\tilde{D}(t, \tau)$ be number of deaths in the population between $t$ and $t+\tau$, mortality term, $\tilde{D} >= 0$.\\
$\tilde{b}(t)=\lim _{\tau \rightarrow 0} \tilde{B}(t, \tau) / \tau$ be numbers of births per unit time.\\
$\tilde{d}(t)=\lim _{\tau \rightarrow 0} \tilde{D}(t, \tau) / \tau$ be numbers of deaths per unit time.
}

\subparagraph{Discrete Time}~{
\begin{equation}
    N(t+\tau)=N(t)+\tilde{B}(t, \tau)-\tilde{D}(t, \tau)\label{eq1.1}
\end{equation}
}

\subparagraph{Continuous Time}~{when $\tau \rightarrow 0$, then
\begin{equation}
    \frac{\mathrm{d} N}{\mathrm{~d} t}=\lim _{\tau \rightarrow 0} \frac{N(t+\tau)-N(t)}{\tau}=\tilde{b}(t)-\tilde{d}(t)\label{eq1.2}
\end{equation}
}

\newpage
\subsection{Single Species Model: Malthusian Model}
\begin{assumption}[Malthusian Assumption]
    The acts of death and birth of animals (vegetation, cell) happen statistically independently of each other and independently of the current population size.
\end{assumption}


\subsubsection{Discrete Time Equation $N(t+\tau)=N(t)+\tilde{B}(t, \tau)-\tilde{D}(t, \tau)$}
Total number of births and deaths happening during timestep $\tau$ proportional to current population size, we have
$$
    \tilde{D}(t, \tau)=D(\tau) N(t), \quad \tilde{B}(t, \tau)=B(\tau) N(t)
$$
then \myref{eq1.1} becomes
\begin{align}
    N(t+\tau) & =(1+B(\tau)+D(\tau))N(t)\notag \\
    N(t+\tau) & =R(\tau)N(t)\label{eq1.3}
\end{align}
where
$$
    R(\tau) = 1+B(\tau)+D(\tau) = \frac{N(t+\tau)}{N(t)}
$$
is called \textbf{\emph{reproduction coefficient}}. $R(\tau)$ is size ratio and only $R(\tau)\geq 0$ make \textbf{biological sense}.$R(\tau) = 0$ is the trivial case: the population dies out within one time step.

\paragraph{Discrete Time Solution}~{}\\
to solve \myref{eq1.3}, we set $N(\tau)=R(\tau)N(0), N(0)\geq 0$ at $t=0$. And for all $t=n\tau$, we have
\begin{align}
     & t=0,N(\tau)=RN(0)\notag                              \\
     & t=\tau, N(2\tau)=R N(\tau)=R^{2} N(0)\notag          \\
     & t=2 \tau, N(3 \tau)=R N(2 \tau)=R^{3} N(0)\notag     \\
     & \qquad \qquad \qquad \quad...\notag                  \\
     & t=n \tau, N(n\tau)=RN(n\tau)=R^{n} N(0)\label{eq1.4}
\end{align}
That is, the population grows (for $R(\tau) > 1$) or decays (for $R(\tau) < 1$) in \emph{\textbf{Geometric Progression}}.

\newpage

\subsubsection{Continuous Time Equation $\frac{\mathrm{d} N}{\mathrm{~d} t}=\tilde{b}(t)-\tilde{d}(t)$}
Now we have the number of births and deaths happening continuously per unit time proportional to current population size.
$$
    \tilde{d}(t)=d N(t), \quad \tilde{b}(t)=b N(t)
$$
then \myref{eq1.2} becomes
\begin{align}
    \frac{\mathrm{d} N}{\mathrm{~d} t} & =\tilde{b}(t)-\tilde{d}(t)\notag \\
    \frac{\mathrm{d} N}{\mathrm{~d} t} & =rN(t) \label{eq1.5}
\end{align}
where
$$
    r=b-d
$$
is called \emph{\textbf{reproduction rate}}, r can be positive or negative.

\paragraph{Continuous Time Solution}~{}\\
We solve \myref{eq1.5} by \textbf{separation of variables}.
\begin{align}
    \frac{\mathrm{d}N}{N(t)}      & =r\mathrm{d}t \notag \\
    \int \frac{\mathrm{d}N}{N(t)} & =\int r d t \notag   \\
    \ln N(t)                      & =rt+C\notag          \\
    N(t)                          & = N(0)e^{rt}
\end{align}
where
$$
    N(t=0)=N(0)\geq 0
$$
This is, the population grows (for $r>0$) or decays (for $r<0$) \emph{\textbf{exponentially}}.

\newpage

\subsubsection{Link of Two Equation}

\paragraph{Discrete $\rightarrow$ Continuous}~{
$
    N(n\tau) = R^n(\tau) N(0)  \rightarrow N(t) = e^{rt}N(0)
$}\\
For small $\tau$,
\begin{align}
    \lim _{\tau \rightarrow 0} \frac{R(\tau)-1}{\tau} & = \lim _{\tau \rightarrow 0}\frac{B(\tau)-D(\tau)}{\tau}\notag \\
    R(\tau)                                           & \approx 1+\tau r\label{eq1.6}
\end{align}
For fixed time t,
$$
    n=\frac{t}{\tau} \rightarrow \infty
$$
Then we have
\begin{align*}
    N(n \tau) & =\lim_{\tau \rightarrow 0}R(\tau)^nN(0)                      \\
              & \approx \lim _{\tau \rightarrow 0}(1+r \tau)^{t / \tau} N(0) \\
              & =e^{rt}N(0)
\end{align*}


\paragraph{Continuous $\rightarrow$ Discrete}~{
$
    N(t) = e^{rt}N(0) \rightarrow N(n\tau) = R^n(\tau) N(0)
$
}\\
Let $t=n\tau$, we have
\begin{align*}
    N(t) & =N(n\tau)                \\
         & =e^{(r\tau)n}N(0)        \\
         & \approx (1+r\tau)^n N(0) \\
         & \approx R^n(\tau)N(0)
\end{align*}
by \myref{eq1.6} as $\tau \rightarrow 0$ and Taylor series.


\newpage
\subsection{Intraspecific Competition Model}

\emph{Competition}: increase of population size suppresses its reproduction.

\begin{assumption}[Intraspecific Competition Assumption]
    Reproduction per capita per unit of time is \textbf{NOT} constant. Instead, reproduction is depends on the current population size.
\end{assumption}

\subparagraph{Discrete Time}~{
    $$
    N_{t+1} = R(N_t)N_t
    $$
    \par{Competition when}
    $$
    dR/dN <0
    $$
}
\subparagraph{Continuous Time}~{
    $$
    dN/dt = r(N)N
    $$
    \par{Competition when}
    $$
    dr/dN <0
    $$
}

\subsubsection{Verhulst (Logistic) Model}
 




\newpage
\subsubsection{Richards Model}



















\newpage


\paragraph{First paragraph}
paragraph

\subparagraph{First subparagraph}
subparagraph

\begin{theorem}[Helge Tverberg 1966]
    Given $(r-1)(d+1)+1$ points in $\rr^d$, there is a partition of them into $r$ parts whose convex hulls intersect.
\end{theorem}



\newpage
\section{3333}


\newpage






\end{document}