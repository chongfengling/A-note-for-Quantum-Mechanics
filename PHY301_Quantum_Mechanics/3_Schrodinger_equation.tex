\section{The Schrödinger equation}

One dimension's Schrödinger equation for wave wave function $\Psi$ ($x \in \mathbf{R}$ is a space coordinate, $t$ is time, $V$ is a \textbf{real-valued potential}) is
\begin{equation}\boxed{
        i \hbar \frac{\partial \Psi(x, t)}{\partial t}=-\frac{\hbar^{2}}{2 m} \frac{\partial^{2}}{\partial x^{2}} \Psi(x, t)+V(x) \Psi(x, t)}\label{eq1}
\end{equation}
Complex conjugate form is
\begin{equation}\boxed{
        -i \hbar \frac{\partial \Psi^{*}}{\partial t}(x, t)=-\frac{\hbar^{2}}{2 m} \frac{\partial^{2}}{\partial x^{2}} \Psi^{*}(x, t)+V(x) \Psi^{*}(x, t)}\label{eq2}
\end{equation}
Planck constant $\hbar$ is
$$
    \hbar=\frac{h}{2 \pi} \simeq 1.05 \times 10^{-34} \mathrm{~kg} \cdot \mathrm{m}^{2} \cdot \mathrm{s}^{-1}
$$
\begin{assumption}\label{ass1}
    $\Psi$ and all its derivatives are smooth and go to zero when $|x|$ goes to infinity, faster than any negative power of $x$.
\end{assumption}

\subsection{Normalization}
Due to its definition, the wave function has to be normalized at all time. That is, for all t, we have
\begin{equation}\boxed{
        \int_{-\infty}^{\infty} \Psi^{*}(x, t) \Psi(x, t) d x=1}\label{eq3}
\end{equation}

\begin{remark}
    Not all solutions of the Schrödinger equation are wave function.
\end{remark}
\begin{theorem}
    A normalized wave function stays normalized.
\end{theorem}
\begin{proof}
    For a normalized wave function at time $t=0$:
    $$
        \int_{-\infty}^{\infty} \Psi^{*}(x, 0) \Psi(x, 0) d x=1
    $$
    consider $t>0$:
    $$
        \begin{aligned}
             & \frac{d}{d t}\left(\int_{-\infty}^{\infty} \Psi^{*}(x, t) \Psi(x, t) d x\right)                                                                                                            \\
             & =\int_{-\infty}^{+\infty} \frac{\partial \Psi^{*}(x, t)}{\partial t} \Psi(x, t) d x+\int_{-\infty}^{+\infty} \Psi^{*}(x, t) \frac{\partial \Psi(x, t)}{\partial t} d x                     \\
             & =\frac{1}{i \hbar} \int_{-\infty}^{+\infty}\left(+\frac{\hbar^{2}}{2 m} \frac{\partial^{2} \Psi^{*}(x, t)}{\partial x^{2}}-V(x) \Psi^{*}(x, t)\right) \Psi(x, t) d x+                      \\
             & \frac{1}{i \hbar} \int_{-\infty}^{+\infty} \Psi^{*}(x, t)\left(-\frac{\hbar^{2}}{2 m} \frac{\partial^{2} \Psi(x, t)}{\partial x^{2}}+V(x) \Psi(x, t)\right) d x                            \\
             & =\frac{\hbar}{2 m i}\left(\left[\Psi \frac{\partial}{\partial x} \Psi^{*}\right]_{-\infty}^{+\infty}-\int \frac{\partial}{\partial x} \Psi^{*} \frac{\partial}{\partial x} \Psi d x\right. \\
             & \left.-\left[\Psi^{*}\frac{\partial}{\partial x} \Psi \right]_{-\infty}^{+\infty}+\int \frac{\partial}{\partial x} \Psi \frac{\partial}{\partial x} \Psi^{*} d x\right)                    \\
             & =0
        \end{aligned}
    $$
    by  Assumption \ref{ass1}. Hence, $\Psi$ is normalized at all time.
\end{proof}

\subsection{Linear momentum}
We can not know actual position of a particle, hence we use the expectation of position $\langle x\rangle(t)$ instead.
\begin{equation}\boxed{
        \langle x\rangle(t)=\int_{-\infty}^{+\infty}  \Psi^{*}(x, t) x \Psi(x, t) d x}\label{eq4}
\end{equation}
the velocity of a particle is
\begin{equation}\boxed{
        \langle V\rangle = \frac{d\langle x\rangle}{d t}}\label{eq5}
\end{equation}
the average linear momentum is
\begin{equation}\boxed{
        \langle p\rangle=\int_{-\infty}^{+\infty} \Psi^{*}(x, t)\left(-i \hbar \frac{\partial}{\partial x}\right) \Psi(x, t) d x}\label{eq6}
\end{equation}
\begin{proof}
    \begin{align*}
        \langle p \rangle & = m \frac{d \langle x \rangle}{d t}                                                                                                                                                                                                                                   \\
                          & = m \int_{-\infty}^{\infty} \frac{d \Psi^*(x,t)}{dt} x \Psi + \Psi^*(x,t) x \frac{d \Psi(x,t)}{dt} dx                                                                                                                                                                 \\
                          & = \frac{m}{i \hbar} \int_{-\infty}^{\infty} (\frac{\hbar^{2}}{2 m} \frac{\partial^{2} \Psi^{*}(x, t)}{\partial x^{2}}-V(x) \Psi^{*}(x, t)) x \Psi(x,t) +    (-\frac{\hbar^{2}}{2 m} \frac{\partial^{2} \Psi(x, t)}{\partial x^{2}} +V(x) \Psi(x, t)) x \Psi^*(x,t) dx \\
                          & = \frac{\hbar}{2i} \intinfty \frac{\partial^{2} \Psi^*(x, t)}{\partial x^{2}} x \Psi(x, t) - \frac{\partial^{2} \Psi(x, t)}{\partial x^{2}}  x \Psi^*(x, t) dx                                                                                                        \\
                          & = \frac{\hbar}{2i} \intinfty x \frac{\partial}{\partial x} (\frac{\partial \Psi^*(x, t)}{\partial x} \Psi(x, t) - \frac{\partial \Psi(x, t)}{\partial x} \Psi^*(x, t)) dx                                                                                             \\
                          & = \frac{\hbar}{2 i} ([x (\frac{\partial \Psi^*(x, t)}{\partial x} \Psi(x, t) - \frac{\partial \Psi(x, t)}{\partial x} \Psi^*(x, t))]_{-\infty}^{+\infty} - \intinfty \frac{\partial \Psi^*(x, t)}{\partial x} \Psi(x, t) - \frac{\partial \Psi(x, t)}{\partial x} \Psi^*(x, t) dx)
                          \\
                          &= \frac{\hbar}{i} \intinfty \frac{\partial \Psi (x,t)}{\partial x} \Psi^* (x,t) dx
                          \\
                          &= \intinfty \Psi^*(x,t) (-i\hbar \frac{\partial}{\partial x}) \Psi(x,t) dx
    \end{align*}
\end{proof}

\subsection{From the Schrödinger equation to the Newton law}
Ehrenfest's theorem
\begin{equation}
    \frac{d\langle p\rangle}{d t}=-\left\langle V^{\prime}(x)\right\rangle
\end{equation}
\boxed{obey\;classical\;laws?}
\begin{proof}
    ... 
\end{proof}


\subsection{Correspondence principle}
\boxed{\hbar\;to\;0?}\\
\textbf{Energy = kinetic energy + potential energy}

\paragraph{Position operator}
...

\paragraph{Linear-momentum operator}
...

\paragraph{Hamiltonian operator}
...

\subsection{Separable solutions: time-independent solutions}
Consider one kind of solutions of Schrödinger equation in the form:
$$
    \Psi(x, t):= \psi(x) \phi(t)
$$
the Schrödinger equation becomes:
\begin{align}
    i \hbar \psi(x) \phi^{\prime}(t)         & =-\frac{\hbar^{2}}{2 m} \psi^{\prime \prime}(x) \phi(t)+V(x) \psi(x) \phi(t)    \\
    i \hbar \frac{\phi^{\prime}(t)}{\phi(t)} & =-\frac{\hbar^{2}}{2 m} \frac{\psi^{\prime \prime}(x)}{\psi(x)}+V(x)\label{eq7}
\end{align}
\myref{eq7} is satisfied for all values of the independent variables $x$ and $t$, hence there exists a \textbf{constant number $E$} that \myref{eq7} $=E$ for both sides. We have
\begin{align}
    i \hbar \frac{\phi^{\prime}(t)}{\phi(t)}                            & =E \label{eq8} \\
    -\frac{\hbar^{2}}{2 m} \frac{\psi^{\prime \prime}(x)}{\psi(x)}+V(x) & =E\label{eq9}
\end{align}
\paragraph{Time-dependent factor \myref{eq8}}~{}
\begin{align}
    \phi^{\prime}(t) &= \frac{E \phi(t)}{i\hbar}\notag\\
    \phi(t) &= \phi(0)e^{\frac{E}{i\hbar}t} \label{eq10}
\end{align}
\paragraph{Space-dependent factor \myref{eq9}}~{}
\begin{align}
    -\frac{\hbar^{2}}{2 m} \psi^{\prime \prime}(x)+V(x) \psi(x)&=E \psi(x)\notag\\
    \hat{H} \psi(x) & = E\psi(x) \label{eq11}
\end{align}
\myref{eq11} is called the \textbf{\emph{time-independent Schrödinger equation}}.\\
There are reasons about why introduce time-independent Schrödinger equation:
1. 
2. 

\begin{remark}
    $E$ is a constant number.
\end{remark}

\begin{remark}
    $E$ is a real number.
\end{remark}
\begin{remark}
    $ \exp(-\frac{iEt}{\hbar})^* = \exp(\frac{iEt}{\hbar})$ when $E \in R $ 
\end{remark}

\begin{proof}
    ...
\end{proof}
\begin{remark}
    $E \geq min(V(x))$
\end{remark}
\begin{proof}
    ...
\end{proof}
\subsection{Example: the infinite square well}
Suppose     
\begin{equation*}
    V(x)= \begin{cases}0, & \text { if } 0 \leq x \leq a \\ \infty, & \text { otherwise }\end{cases}
    \end{equation*}
then $\psi(x)=0$ when $x \notin [0,a]$. When $x \in [0,a]$, we have time-independent Schrödinger $\psi(x)$ satisfied
\begin{align}
    -\frac{\hbar^{2}}{2 m} \frac{d^{2} \psi}{d x^{2}}&=E \psi \notag\\
    \frac{d^{2} \psi}{d x^{2}}&=-k^{2} \psi \label{eq12}
\end{align}
where 
\begin{align}
    \boxed{k \equiv \frac{\sqrt{2 m E}}{\hbar}} \label{eq13}
\end{align}
\myref{eq12} is the classical simple harmonic oscillator equation with general solution
\begin{align}
    \psi(x)=A \sin k x+B \cos k x
\end{align}
where $A$ and $B$ are constant.\\
Continuity of $\psi(x)$ requires boundary condition
$$\psi(0)=\psi(a)=0$$
by $\psi(0)=0$, we have $B=0$, $\psi(x)=A \sin kx$\\
by $\psi(a)=0$,, we have trivial (non-normalizable) solution $A=0$ or $\sin kx=0$, \\$k a=0, \pm \pi, \pm 2 \pi, \pm 3 \pi, \ldots$, $k\leq 0$ is useless. Hence the distinct solutions are 
$$
k_{n}=\frac{n \pi}{a}, \quad \text { with } n=1,2,3, \ldots
$$
According to \myref{eq13}, the corresponding values of $E$ is 
$$
E_{n}=\frac{\hbar^{2} k_{n}^{2}}{2 m}=\frac{n^{2} \pi^{2} \hbar^{2}}{2 m a^{2}}
$$
To find $A$, we normalize $\Psi(x,t)$
\begin{align*}
    \intinfty |\Psi(x,t)|^2 dx &= 1\\
    \intinfty \psi(x)^2 dx &=1\\
    \intinfty |A|^2 sin^2(kx) dx &= 1\\
    |A|^2 &=\frac{2}{a}\\
    A &= \sqrt{\frac{2}{a}}
\end{align*}

Finally, we have solutions of time-independent Schrödinger \myref{eq11}
$$
\psi_{n}(x)=\sqrt{\frac{2}{a}} \sin \left(\frac{n \pi}{a} x\right)
$$
with corresponding energy
$$
E_{n}=\frac{n^{2} \pi^{2} \hbar^{2}}{2 m a^{2}}
$$




\paragraph{Consequence I}~{
    if the wave function at $t=0$ is a normalized eigenfunction $\psi(x)$ (ie. $n=1$), we integrate the Schrödinger equation by introducing a phase factor, we have
$$
\Psi(x,t) = \exp(-\frac{iEt}{\hbar})\psi(x), \forall x, \forall t.
$$
$|\Psi(x,t)^|2$ and then $\langle x\rangle$ are independent on time $t$. The wave function is stationary.
}

\paragraph{Consequence II}~{
If the wave function at $t=0$ is a sum of eigenfunction $\psi(x)$ with $c_n \in \mathbb{C} $, we integrate the Schrödinger equation by introducing phase factors, we have
$$
\Psi(x,t) = A \exp{-\frac{iE_1 t}{\hbar}}\psi_1(x) + B \exp{-\frac{iE_2 t}{\hbar}}\psi_2(x), A \in \mathbb{C},\forall x, \forall t
$$
if $E_1\neq E_2$, then $|\Psi(x,t)|^2$ and then $\langle x\rangle$ are dependent on time $t$. The wave function is a periodic function of time.
}






\subsection{The free particle: time-dependent solutions}
When $V(x)=0$, we have 
\begin{align*}
    i \hbar \frac{\phi^{\prime}(t)}{\phi(t)}                            & =E  \\
    -\frac{\hbar^{2}}{2 m} \frac{\psi^{\prime \prime}(x)}{\psi(x)} & =E
\end{align*}
by \myref{eq8}, \myref{eq9}. let $k \equiv \frac{\sqrt{2 m E}}{\hbar} > 0$, we get general solution
\begin{align*}
    \Psi(x, t)=A e^{i k\left(x-\frac{\hbar k}{2 m} t\right)}+B e^{-i k\left(x+\frac{\hbar k}{2 m} t\right)}
\end{align*}
let $k \equiv \pm \frac{\sqrt{2 m E}}{\hbar}$
the solution becomes
\begin{align}
    \Psi_{k}(x, t)=A e^{i\left(k x-\frac{\hbar k^{2}}{2 m} t\right)} \label{eq14}
\end{align}
\myref{eq14} has a space period $\lambda$ and a time period $T$.\\
\myref{eq14} is not normalizable.






























% \subsection{Example: Gaussian wave packet}
% the wave function is
% $$
%     \Psi(x, t)=\frac{1}{(2 \pi)^{\frac{1}{4}} \sqrt{\sigma}\left(1+\frac{i \hbar t}{2 m \sigma^{2}}\right)^{\frac{1}{4}}} \exp \left(-\frac{x^{2}}{4 \sigma^{2}\left(1+\frac{i \hbar t}{2 m \sigma^{2}}\right)}\right)
% $$
% calculate $\Psi(x,t)^*\Psi(x,t)$, we have
% \begin{align*}
%     \Psi(x,t)^*\Psi(x,t) & =  \frac{1}{(2 \pi)^{\frac{1}{2}} \sigma \left(\left(1+\frac{i \hbar t}{2 m \sigma^{2}}\right) \left(1-\frac{i \hbar t}{2 m \sigma^{2}}\right)\right)^{\frac{1}{4}}} \exp(-\frac{x^2}{4\sigma^2} (\frac{1}{1+\frac{i \hbar t}{2 m \sigma^2}}+\frac{1}{1-\frac{i \hbar t}{2 m \sigma^2}})) \\
%                          & =  \frac{1}{(2 \pi)^\frac{1}{2} \sigma (1+(\frac{\hbar t}{2 m \sigma^2})^2)^\frac{1}{4}} \exp(-\frac{x^2}{2 \sigma^2 (1+(\frac{\hbar t}{2 m \sigma^2})^2)})                                                                                                                               \\
%                          & = \frac{\sqrt{w}}{\sqrt{\sigma \pi}} \exp(-2 w^2 x^2)
% \end{align*}
% in terms of the quantity $w = \frac{1}{2 \sigma} \frac{1}{\sqrt{1+(\frac{\hbar t}{2 m \sigma^2})^2}}$
