\section{Wave Function}
\subsection{Definition in 1-D space}
\begin{definition}[Wave Function]
    For a small particle living in a \textbf{{one-dimensional space}}, the wave function $\Psi$ is a complex-valued function of space and time:
    $$
        \begin{gathered}
            \Psi: \mathbf{R} \times \mathbf{R} \rightarrow \mathbb{C} \\
            (x, t) \mapsto \Psi(x, t) \in \mathbf{C}
        \end{gathered}
    $$
\end{definition}

\begin{remark}
    $
        |\Psi(x,t)|^2
    $
    is the p.d.f. of finding the particle in position x at time t.
\end{remark}
\begin{remark}
    $
        \int_{a}^{b} |\Psi(x,t)|^2 dx
    $
    is the c.d.f of finding the particle between position $[a, b]$ at time t.
\end{remark}
\begin{remark}
    Integration is over \textbf{space}, t is a \textbf{parameter}.
\end{remark}
\newpage
\subsection{Mean and variance of the position}
These two statistics are expressed as integrals over the entire space. They are \textbf{deterministic functions of time}. Given wave function $\Psi$, then we have: 
$$
\begin{gathered}
    \langle x\rangle(t)=\int_{-\infty}^{+\infty} x|\Psi(x, t)|^{2} d x\\
    \langle Var(x)\rangle(t)=\int_{-\infty}^{+\infty}(x-\langle x\rangle(t))^{2}|\Psi(x, t)|^{2} d x\\
    [\Psi] = \frac{1}{\sqrt{L}}
\end{gathered}
$$

\subsection{Example: probability density of position for classical object}
...